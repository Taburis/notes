\section{Manifolds}



\subsection{Differetiable Manifolds}
\begin{definition}
A $m$-dimensional \textbf{manifold} or \textbf{topological manifold} is a Hausdorff spacce $M$ that $\forall x\in M$, $\exists U_x$ a neighborhood of $x$ that $U_x$ is homeomorphic to an open set in $\realR^m$. Suppose the homeomorphism is $\varphi_U:U\to\realR^m$, then $(U,\varphi_U)$ is called the \textbf{coordinate chart} of $M$. And the mapping $u=\varphi_U(y)$, $u=(u_1,\dots,u_m)$ is called the \textbf{local cooridnate} of the point $y\in U$. Two cooridnate charts $(U,\varphi_U)$ and $(V,\varphi_V)$ are called $C^r$-\textbf{compatible} if $\forall p\in U\cap V$:
\begin{equation}
\begin{aligned}
&g(f(x))=x,\quad x=\varphi_U(p),\\
f&=\varphi_U\circ\varphi_V^{-1}:\varphi_V(U\cap V)\to \varphi_U(U\cap V),\\
g&=\varphi_V\circ\varphi_U^{-1}:\varphi_U(U\cap V)\to \varphi_V(U\cap V),
\end{aligned}
\end{equation}
and $f,g\in C^r$ ($r$-th order of partial derivative exists and continuous). 
\end{definition} 

\begin{definition}
A $C^r$-\textbf{differentiable structure} on a manifold $M$ is a coordinate charts $\mathcal{A}=\{(U,\varphi_U),(V,\varphi_V),\dots\}$ satisfying the following conditions
\begin{enumerate}
\item $\{U,V,\dots\}$ forms an open covering of $M$;
\item Any two coordinate charts are $C^r$-compatible to each other;
\item $\mathcal{A}$ is the \textbf{maximal}, that is, if a $(\alpha,\varphi_\alpha)$ is a coordinate chart that $C^r$-compatible with all the charts in $\mathcal{A}$, then $(\alpha,\varphi_\alpha)\in\mathcal{A}$.
\end{enumerate}
A manifold $M$ is called $C^r$\textbf{-differentiable manifold} if there's a $C^r$-differentiable structure given on $M$. A $f\in C^\infty$ is usually called \textbf{smooth}. So $C^\infty$-differentiable manifold is called smooth manifold.
\end{definition}

The manifold is \textbf{locally compact} since any manifolds are homeomorphic to $\realR^m$ which is locally compact. It also shows that the properties of $\realR^m$ preserved by homeomorphism is also the properties of the manifold. 

\begin{lemma}
Suppose $(U,\varphi_U)$ is a coordinate chart in a smooth manifold $M$, $V\ne\varnothing$ is an open set in $M$ with $\overline{V}$ compact, and $\overline{V}\subset U$. Then there exists a smooth function $h:M\to\realR$ such that
\begin{enumerate}
\item $0\le h\le 1$;
\item $h(p)$ has the properties:
\begin{equation}
h(p)=\left\{
\begin{aligned}
&0,\quad p\in V,\\
&1,\quad p\notin U.
\end{aligned}
\right.
\end{equation}
\end{enumerate}
\end{lemma}

\begin{proof}
We need to prove this lemma in three steps.
Suppose $B_a(0),B_b(0)$ are balls at the origin in $\realR^m$ with radius $a<b$, respectively. Then we can construct a function $g$ as follow:
\begin{equation}
g(x)=\left\{
\begin{aligned}
&\exp\Bigg\{\frac{1}{(x-a^2)(x-b^2)}\Bigg\},&\quad x\in (a^2,b^2),\\
&0 ,&\quad \text{otherwise}.
\end{aligned}
\right.
\end{equation}
Furthermore, the $f(x)$ defined as
\begin{equation}
f(x)=\frac{\int_x^{+\infty} g(t)dt}{\int_{-\infty}^{+\infty}g(t)dt},
\end{equation}
are continuous and satisfy the conditions that $f(x)=1,\forall x\in B_a(0)$ while $f(x)=0,\forall x\notin B_b(0)$.
Futhermore, let $V'\subset U'$ are both open sets in $\realR^m$, and $\overline{V'}$ is compact, then there exists an open cover $\{B_a(x),B_b(x):x\in V\}$ which $\cup B_b(x)\subset U'$ and $\overline{V'}\subset \cup B_a(x)$. The compactness implies there is finite subcover $\{\{B_a(x_i),B_b(x_i):1\le i\le r\}$ and the function $f_i$ is the function defined previous corresponds to the $i$th ball. Then the function
\begin{equation}
F=1-\prod_i(1-f_i),
\end{equation}
satisfies the requirement that $F(x)=1,\forall x\in V$ and $F(x)=0,\forall x\notin U$. To extend this lemma to the smooth manifold, the local compactness makes sure that there is a open set $W$ such that $\overline{V}\subset W\subset\overline{W}\subset U$. The mapping $\varphi_U(W)=U'$ and $\varphi_U(V)=V'$, then the function $h(p)=F\circ\varphi_U$ is the function in the lemma.
\end{proof}



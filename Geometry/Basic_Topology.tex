\section{Topological Spaces}
\begin{definition}
A \textbf{Topological space} is a pair $(H,\mathfrak{F})$ where $H$ is a set and $\mathfrak{F}$ is the collection of the subsets of $H$ satisfying the following axioms:
\begin{enumerate}
\item $\varnothing\in\mathfrak{F}$;
\item Any arbitrary union of members of $\mathfrak{F}$ stills belong to $\mathfrak{F}$;
\item The intersection of any finite number of members of $\mathfrak{F}$ still belongs to $\mathfrak{F}$.
\end{enumerate}
\end{definition}

\begin{definition}
For a topological space $(H,\mathfrak{H})$:
\begin{itemize}
\item The sets in $\mathfrak{F}$ are called \textbf{open set}, the complement of a open set is called \textbf{closed set}.
\item A open cover for $X$ is sequence of open sets $U_{i\in\mathcal{A}}\in\mathfrak{F}$ that
\begin{equation}
X\subseteq \bigcup_{i\in \mathcal{A}}U_i.
\end{equation}
If the number of the covers is finite, then it called a finite cover of $X$.
\item A set $V$ is called a \textbf{compact set}, if for any open cover $U_{i\in\mathcal{A}}$ of $V$, there's a finite subsequence of $U_{i\in\mathcal{A}}$ formed a finite cover of $V$. 
\item A \textbf{neighbourhood} of $p\in H$ is a subsect $V\subseteq H$ that $\exists U\in\mathfrak{H}:p\in U\subset V$.
\item A \textbf{limit point} of $U\subset H$ is a point $p\in H$ that $\forall U_x\in\mathfrak{H}: U_x\cap U\ne\varnothing$. 
\item Space $H$ is called \textbf{locally compact} if $\forall p\in H, \exists V:p\in U_x\subseteq V$ where $U_x$ is an open set and $V$ is a compact neighourhood of $p$.
\item A \textbf{Closure} $\overline{U}$ of $U\subseteq H$ is the set contained all the limit points of $U$.
\item A \textbf{Boundary} of $U$ is a set $\partial U:=\overline{U}\cap\overline{H\setminus U}$.
\item A \textbf{basis} of $H$ is a subcollection $\mathfrak{G}\subset\mathfrak{H}$ that for any open set, it can be expressed as the union of open sets in $\mathfrak{G}$. Clearly the base formed an open cover of $H$.
\end{itemize}
\end{definition}

\begin{definition}
Suppose $f:X\to Y$ is a mapping from topological space $(X,\mathfrak{R})$ to $(Y,\mathfrak{F})$, then
\begin{itemize}
\item A \textbf{continuous mapping} if $\forall V\in \mathfrak{F}: f^{-1}(V)\in\mathfrak{R}$. 
\item A \textbf{homeomorphism mapping} is a one-to-one continuous mapping $f$ that the inverse $f^{-1}$ is continuous. All of the self-homeomorphisms $f:X\to X$ formed a group, denoted as $\text{Homeo}(X)$. 
\end{itemize}\index{Continuous function}
\end{definition}

\begin{theorem}
For a topologic space $X$ and a mapping $f$, the follow statments are equivalent:
\begin{enumerate}
\item $f$ is continuous,
\item For any closed set $V\subset Y$, $f^{-1}(V)$ is closed;
\item For any $x\in X$ and every neighbourhood $V_{f(x)}$ of $f(x)$, there is a neighbourhood $U$ of $x$ that $f(U)\subseteq V$;
\item $\forall A\subseteq X$, $f(\overline{A})\subseteq \overline{f(A)}$;
\item $\forall B\subseteq Y$, $\overline{f^{-1}(B)}\subseteq f^{-1}(\overline{B})$.
\end{enumerate}
\end{theorem}

\begin{proof}
To prove the sufficiency of the 2nd point, for any close set $V$, the preimage $f^{-1}(X\setminus V)$ is open and $f^{-1}(X\setminus V\cup V)=f^{-1}(X\setminus V)\cup f^{-1}(V)$ which means that $f^{-1}(V)$ is closed. The similar reason can leads to the necessarity. 
$1\to 3$: is obviously by setting $U=f^{-1}(V)$.

$3\to 4$: Suppose $x\in \overline{A}$ and $U_x$ is an arbitrary neighbourhood of $x$ and the neighbourhood $V_{f(x)}$ of $f(x)$, $f(U_x\cap A)\subset f(U_x)\cap f(A)\subseteq V_{f(x)}\cap f(A)\ne\varnothing$ which implies $f(x)\in \overline{A}$.

$4\to 5$: Since $f(\overline{A})\subseteq \overline{f(A)}=\overline{f[f^{-1}(B)]}$ where $A=f^{-1}(B)$, it implies $\overline {A}=\overline{f^{-1}(B)}\subseteq f^{-1}(\overline{B})$.

$5\to 2$: Let $B$ be a closed set and since $f^{-1}(B)\subseteq \overline{f^{-1}(B)}$ which implies $\overline{f^{-1}(B)}=f^{-1}(B)$ and hence $f^{-1}(B)$ is closed.
\end{proof}

\begin{definition}(Separation axioms:)
Suppose $(X,\mathfrak{F})$ is a topological space,
\begin{itemize}
\item $X$ is $T_0$, or \textbf{Kolmogorov}, if any two distinct points in $p\ne q,p,q\in\mathfrak{F}: \exists U_p,U_q\in\mathfrak{F}$ that $U_p\ne U_q$.
\item $X$ is $T_1$, or \textbf{Fr\'{e}chet} or \textbf{Tikhonov},  
\end{itemize}
\end{definition}

\begin{theorem}
Any compact set $U$ in Hausdorff space $(H,\mathfrak{H})$ is closed (contained all the limit points of itself).
\end{theorem}

\begin{proof}
To proof $U$ is closed, we are going to show $H\setminus U$ is open which means
\begin{equation}
\forall x\in H\setminus U, \exists V_x\in\mathfrak{H}: V_x\subseteq H\setminus U.
\end{equation}
In fact
\begin{equation}
\forall x\in H\setminus U,\,\forall y\in U,\,\exists (U_y,V_y^x)\in\mathfrak{H}: x\in V_y^x,\,U_y\cap V_y^x=\varnothing
\end{equation}
and $\{U_y:y\in U\}$ formed an open cover of $U$ and the compactness requred that a finite subset $F\subseteq U$ that $\{U_y:y\in F\}$ formed an finite open cover of $U$. However, 
\begin{equation}
V=\bigcap_{y\in F}V_y^x
\end{equation}
is a open set that $x\in V$ but $V\cap U=\varnothing$.
\end{proof}

\begin{theorem}
If $X$ is a locally compact Hausdorff space, then for any neighborhood $U_x$ of $x$, there exists a neighberhood $V$ such that $x\in V\subseteq\bar V\subseteq U$ and $\bar V$ is compact.
\end{theorem}
\begin{proof}
The idea is that there should be an open covers ${W_i}$ for the boundary $\partial U$ and a neighborhood $V$ such that $\forall W_i: V\cap\notin W_i=\varnothing$ (Hausdorff). This implies $\bar V\cap \partial U=\varnothing$.  
\end{proof}
\section{Quantum Mechanics}

\subsection{Wave Function}

According to the central limit theory, the measurement of a particle at position $x=0$ would leads to a Gaussian distribution around $x=0$ with a uncertainty, say $\sigma$, then the wave function for this distribution has to be 
\begin{equation}
\varphi(x)\sim \exp\left(-\frac{x^2}{4\sigma^2}\right),
\end{equation}
so that $|\varphi(x)|^2\sim e^{-\frac{x^2}{2\sigma^2}}$. Then the wave function in momentum space is
\begin{equation}
\begin{aligned}
\varphi(k)&\sim\int_{-\infty}^{+\infty}\exp\left(-\frac{x^2}{4\sigma^2}\right)\cdot\exp\left\{-\frac{i}{\hbar}xk\right\}dk,\\
&\sim \exp\left\{-\frac{k^2\sigma^2}{\hbar^2}\right\},
\end{aligned}
\end{equation}
where we used the formula of \textbf{Hubbard-Stratonovich} transformation
\begin{equation}
\exp\left(-\frac{\alpha k^2}{2}\right)=\sqrt{\frac{1}{2\pi\alpha}}\int_{-\infty}^{+\infty}\exp\left(-\frac{x^2}{2\alpha}-ikx\right)dx.
\end{equation}
The distribution of the momentum measurement is $|\varphi(k)|^2\sim \exp\{-2k^2\sigma^2/\hbar^2\}$, and the variation is $\sigma_k^2= 4\hbar^2/\sigma^2$ which means that it satisfies the lower bound of the uncertainty principle $\sigma_x\sigma_k=\hbar/2$. To minimize the variation, we require that $\sigma_k=\sigma_x=\sigma$ and it would leads to $\sigma=\sqrt{\hbar/2}$ and the $\varphi(x)=\exp\{-\frac{x^2}{2\hbar}\}$.


The Born interpretation suggests that a wave function $|\varphi(\boldsymbol{x},t)|^2$ should be treated as the probability density at the position $(\boldsymbol{x},t)$ and this leads to a requirement
\begin{equation}
\int_V |\varphi(\boldsymbol{x},t)|^2d^3x=\text{Constant}.
\end{equation}
With this, we can normalize this integral to 1. This conservation can be extended as
\begin{equation}
\begin{aligned}
\partial_\mu J^\mu(x) &= 0,\\ 
J^\mu(x) &= (c\rho,\boldsymbol{J}),\\
\rho(x)&=|\varphi(x)|^2,\\
\boldsymbol{J}(x)&=\rho(x)\boldsymbol{v}(x).
\end{aligned}
\end{equation}


\begin{theorem}
Satisfying the Schr\"{o}dinger's equation
\begin{equation}
i\hbar\partial_0\psi=H\psi,
\end{equation}
is a sufficient condition for a wave function to be probability conserved.
\end{theorem}
\begin{proof}
In fact, if $\psi$ satisfies the Schr\"{o}dinger's equation, then
\begin{equation}
i\hbar\frac{d}{dt}\int_V|\psi(x)|^2d^3x=i\hbar\int_V\left\{\psi^*(H\psi)-(H\psi)^*\psi\right\}d^3x,
\end{equation}
and this integral vanishes as long as $H$ is a Hermitian operator.
\end{proof}



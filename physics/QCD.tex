\section{Quantum Chromodynamics}

\subsection{Parton Model}

\subsubsection{Deep Inelastic Scattering}
The experiment facts shows that in the proton-proton collisions, the large transverse momentum events are suppressed and most of the poins are produced collinear to the collision axis. Particles produced with a large trnasverse momentum is a consequence from a large $q^2$, momentum transfer between the protons. The limited transverse momentum implise a hyperthesis that proton is a loose bound system, like jelly, which are not able to absrob a large $q^2$. However, this hyperthesis contradicted to the deep inelastic scattering at SLAC-MIT experiment which observed a substantial rate of hard scattering of electrons from the protons in hydrogen target. This leads to a \textbf{parton model} (Bjorken, Feynman) that proton is consisted by several point-like particles called \textbf{partons}. These partons have the electromagnetic interactions which means they should have some electric charge. Then there's an strong interactions bound them together to prevent any one of them been push out from proton due to the electric interaction within themselves. 


To formulate the parton model, the scattering of two point-like particles in massless limit (which is a good approximation in high energy case that $E\gg m$) is characterised by
\begin{equation} 
\frac{1}{4}\sum_{\text{spin}}|\mathcal{M}|^2=\frac{8e^4}{\hat t^4}\left(\frac{\hat s^2}{4}+\frac{\hat u^2}{4}\right),
\end{equation}
where the $s,t$ and $u$ are Mandelstam variables. Then the cross section is
\begin{equation}
\frac{d\sigma}{d\hat t}=\frac{2\pi\alpha^2Q_i^2}{\hat s^2}\left(\frac{\hat s^2+(\hat s+\hat t)^2}{\hat t}\right),\label{eq:2to2_point-like_massless_limit}
\end{equation}
where is the electric charge of parton in the unit of $e$. To characterise the parton inside the proton, let $f_i(x)$ be a function of probability of a type $i$ parton inside the proton carrying a longitudinal fraction $x$ ($0\le x\le 1$) of the total momentum $P$ of the proton. And considering the scattering of $e_{k}Q_{p}\to e_{k'}Q_{p+q}$ where $Q$ is a parton in the proton with momentum $P$. 
\begin{equation}
\begin{aligned}
\hat s&= (p+k)^2=2p\cdot k =2xP\cdot k=xs,\\
x &\approx \frac{Q^2}{2P\cdot q},\quad Q^2=-q^2,
\end{aligned}
\end{equation}
where $s$ is the square of the electron-proton center of mass energy and the last approximation comes from the assumption that $(p+q)^2\approx0$, the mass of the parton is negligible. $q^2$ is space-like to produce a large transverse momentum final state so that $Q^2$ is positive. Then the cross section of a electron scattering from a parton inside a proton is just a scaled up of the original Eq.~\ref{eq:2to2_point-like_massless_limit}, two point-like particle scattering, into
\begin{equation}
\frac{d^2\sigma}{dxdQ^2}=\sum_if_i(x)Q^2_i\cdot\frac{2\pi\alpha^2}{Q^4}\left[1+\bigg(1-\frac{Q^2}{xs}\bigg)^2\right].
\end{equation}
This scaling is called \textbf{Bjorken scaling}.
\section{Functional Analysis}

\subsection{Measure Theory}

\subsubsection{Collections of sets}

\begin{definition}
A collection of sets $\mathfrak{S}$ is called a \textbf{semi-ring} if
\begin{enumerate}
\item $\varnothing\in\mathfrak{S}$;
\item $\forall A,B\in\mathfrak{S}:\,A\cap B\in\mathfrak{S}$;
\item If $A, A_1\in\mathfrak{R}$ and $A_1\subseteq A$, then $\exists A_i\in\mathfrak{S},i=2,\dots,n$ so that
\begin{equation}
A=\bigcup^n_{i=1} A_i,\quad \forall i\ne j:A_i\cap A_j=\varnothing.
\end{equation} 
\end{enumerate}
\end{definition}


\begin{definition}
A collection of sets $\mathfrak{R}$ is called a \textbf{ring} if $\forall A,B\in\mathfrak{R}$:
\begin{enumerate}
\item $A\cap B\in\mathfrak{R}$,
\item $A\triangle B\in\mathfrak{R}$.
\end{enumerate}
\end{definition}

A ring $\mathfrak{R}$ is also closed for the operation union and difference since
\begin{equation}
\begin{aligned}
A\cup B&=(A\triangle B)\triangle(A\cap B)\\
A\setminus B&=A\cap(A\triangle B),
\end{aligned}
\end{equation}
and it implies that $\varnothing\in\mathfrak{R}$ since $A\setminus A=\varnothing$.

\begin{theorem}
The intersection of arbitrary number of rings $\mathfrak{R}=\cap_i\mathfrak{R}_i$ is a ring.
\end{theorem}

\begin{theorem}
If $\mathfrak{S}$ is an arbitrary non-empty set, then there exists precisely one ring $\mathfrak{R}(\mathfrak{S})$ containing $\mathfrak{S}$ and contained by every rings containing $\mathfrak{S}$.
\end{theorem}

\begin{lemma}\label{lemma:semi-ring_1}
Given a sequence of pair-wise disjoint subsets $A_1,\dots,A_m\in\mathfrak{S}$, then there exists another sequence $A_{m+1},\dots,A_n\in\mathfrak{S}$ such that
\begin{equation}
A=\bigcup_{i=1}^nA_i,\quad A_i\cap A_j=\varnothing\text{ for }i\ne j. 
\end{equation}
\end{lemma}

\begin{proof}
The proof is based on the induction method, and the $n=1$ case valid by the definition of semi-ring. Suppose the lemma satisfies for $m=M$, and we will prove it valid for $M+1$ case. In fact, let $S_{M}=\sum_{i=1}^MA_i$, then $S_M\subseteq A$ which means that $\exists B_j,j=1,\dots,B_s$ such that
\begin{equation}
A=S_M\cup\sum_{j=1}^sB_j,\quad B_i\cap B_j\text{ for }i\ne j.
\end{equation}
Now, we set $B_{1,j}=A_{M+1}\cap B_j$, then we have a pair-wise disjointed sequence of $B_{k,j}$ such that
\begin{equation}
B_i=\bigcup_{k=1}^{n_i}B_{k,i},
\end{equation}
Then we got finally
\begin{equation}
A=\left(\bigcup_{i=1}^{M+1}A_i\right)\cup\left(\bigcup_{j=1}^s\bigcup_{k=2}^{n_j}B_{k,j}\right).
\end{equation}
\end{proof}

\begin{lemma}\label{lemma:semi-ring_2}
If $A_1,\dots,A_n$ are elements of a semi-ring $\mathfrak{S}$, there exists a finite set of pair-wise disjoint sets $B_1,\dots,B_m$ such that for any $A_k$:
\begin{equation}
A_k=\bigcup_{i\in M_k}B_i,
\end{equation}
where $M_k$ is the indices combination.
\end{lemma}

\begin{proof}
The lemma is valid for $n=1$ case, by induction, we assume that the lemma is valid for $n=m$ with $B_1,\dots,B_t$, and consider the $m+1$ case. Let
\begin{equation}
B_{1s}=A_{m+1}\cap B_s,\quad s=1,\cdots,t
\end{equation} 
and, by the previous lemma, we have
\begin{equation}
A_{m+1}=\left(\bigcup_{s=1}^tB_{1s}\right)\cup\left(\bigcup_{j=1}^hB'_j\right),
\end{equation}
On the other hand, $B_{1s}\subseteq B_s$ and leads to 
\begin{equation}
B_s=\bigcup_{i=1}^{n_s}B_i,
\end{equation}
which leads to the final expansion
\begin{equation}
A_k=\bigcup_{s\in M_k}\bigcup_{\alpha=1}^{n_s}B_{\alpha s}.
\end{equation}
And $B_s,B'_j$ are pair-wise disjoint.
\end{proof}


\begin{lemma}\label{lemma:extension_of_semi-ring}
Given a semi-ring $\mathfrak{S}$, the ring $\mathfrak{R}(\mathfrak{S})$ generated by $\mathfrak{S}$ is coincides with the collection $\mathfrak{B}$ of the sets $A$ which admit of a finite partition
\begin{equation}
A=\bigcup_{i=1}^nA_i,\quad A_i\in\mathfrak{S}.
\end{equation} 
\end{lemma}
\begin{proof}
To show the $\mathfrak{B}$ is a ring, we suppose $A,B\in\mathfrak{B}$ that 
\begin{equation}
A=\bigcup_{i=1}^nA_i,\quad B=\bigcup_{j=1}^mB_j,\quad A_i,B_j\in\mathfrak{S},
\end{equation}
and let $C_{ij}=A_i\cap B_j$, then, by the lemma~\ref{lemma:semi-ring_1}, we have
\begin{equation*}
A_i=\left(\bigcup_{j}C_{ij}\right)\cup\left(\bigcup_{k=1}^{n_i}\tilde A_k\right),\quad B_j=\left(\bigcup_{i}C_{ij}\right)\cup\left(\bigcup_{l=1}^{m_j}\tilde B_l\right)
\end{equation*}
where $\tilde A_k,\tilde B_l\in\mathfrak{S}$ and pair-wise disjointed with the $C_{ij}$. So that
\begin{equation}
\begin{aligned}
A\cap B=&\bigcup_{ij}^{n,m}C_{ij},\\
A\triangle B = &\left(\bigcup_{k=1}^{n_i}\tilde A_k\right)\cup\left(\bigcup_{l=1}^{m_j}\tilde B_l\right).
\end{aligned}
\end{equation}
which proved that $\mathfrak{B}$ is a ring. Obviously, it is the minimal ring that contained the $\mathfrak{S}$ and hence, it coincides with $\mathfrak{R}(\mathfrak{S})$.
\end{proof}

\begin{definition}
A ring $\mathfrak{R}$ of sets is called \textbf{$\sigma$-ring} if the union $S=\cup_iA_i$ of any countable sets $A_i\in\mathfrak{R}$ implies $S\in\mathfrak{R}$.

A ring $\mathfrak{R}$ of sets is called \textbf{$\delta$-ring} if the intersection $D=\cap_iA_i$ of any countable sets $A_i\in\mathfrak{R}$ implies $D\in\mathfrak{R}$.

A ring with unit is called an \textbf{algebra}.
\end{definition}

\begin{theorem}
$\delta$-algebra, a $\delta$-ring with a unit, is also a $\sigma$-algebra, a $\sigma$-ring with unit. So we called themm \textbf{Borel-algerbra}.
\end{theorem}
\begin{proof}
It is a following conclusion of the duality equations
\begin{equation}
\begin{aligned}
\bigcup_iA_i=&E\setminus \bigcap_i(E\setminus A_i)\\
\bigcap_kA_k=&E\setminus \bigcup_i(E\setminus A_i).
\end{aligned}
\end{equation}
\end{proof}

\begin{theorem}
Given a collection of sets $\mathfrak{S}$, there exists a Borel-algebra $\mathfrak{B}(S)$ containing the $\mathfrak{S}$ and contained in every Borel-algebra containing the $\mathfrak{S}$.
\end{theorem}

A Borel-algebra $\mathfrak{B}(\mathfrak{S})$ is called the minimal $B$-algebra over the system $\mathfrak{S}$ or the \textbf{Borel closure} of $\mathfrak{S}$.

\begin{theorem}
Let $f:M\to N$ be a function and $\mathfrak{M}$ be a collection of subsets of $M$. Then $f(\mathfrak{M})$ denotes the collection of all images $f(A),A\in\mathfrak{M}$. Similarly, $f(\mathfrak{N})$ denotes the collection of all inverse images $f^{-1}(A),A\in\mathfrak{N}$. Then we have
\begin{itemize}
\item If $\mathfrak{N}$ is a ring, then $f^{-1}(\mathfrak{N})$ is a ring.
\item If $\mathfrak{N}$ is an algerbra, then $f^{-1}(\mathfrak{N})$ is an algebra.
\item If $\mathfrak{N}$ is a $B$-algerbra, then $f^{-1}(\mathfrak{N})$ is a $B$-algebra.
\item $\mathfrak{R}(f^{-1}(\mathfrak{N}))=f^{-1}(\mathfrak{R}(\mathfrak{N}))$.
\item $\mathfrak{B}(f^{-1}(\mathfrak{N}))=f^{-1}(\mathfrak{B}(\mathfrak{N}))$.
\end{itemize}
\end{theorem}


\subsubsection{Mesurement on semi-ring and extension to rings}

\begin{definition}
A function $\mu:\mathfrak{S}\to\realR^+\cup\{0\}$, where $\mathfrak{S}$ is a collection of sets, is called a \textbf{measure} of sets if
\begin{enumerate}
\item $\mathfrak{S}$ is a semi-ring.
\item It is additive, that is, if $A=\cup_{i=1}^nA_i$ where $A_i\in \mathfrak{S}$ is a finite partition of $A$, then
\begin{equation}
\mu(A)=\sum_{i=1}^n\mu(A_i).
\end{equation}
\end{enumerate} 
\end{definition}

Based on the additive property, $\mu(\varnothing)=0$. 
\begin{theorem}
Suppose $A^\circ_i$ are finite pair-wise disjoint subset of $A$, then
\begin{equation}
\mu(A)\ge \sum_{i=1}^nA^\circ_i.
\end{equation}
Suppose $A_i$ are any finite sets in $\mathfrak{S}$ such that $A\subseteq \cup_{j=1}^mA_j$, then
\begin{equation}
\mu(A)\le\sum_{j=1}^mA_j.
\end{equation}
\end{theorem}

\begin{proof}
To prove the first inequality, the lemma~\ref{lemma:semi-ring_1} shows that $A=(\cup_{i=1}A^\circ_i)\cup (\cup_{k=1}^{h}\tilde A_k)$, then the inequality is conclude since
\begin{equation*}
\mu(A)=\sum_{i=1}^n\mu(A^\circ_i)+\sum_{k=1}^h\mu(\tilde A_k)\ge\sum_{i=1}^n\mu(A^\circ_i).
\end{equation*}
To prove the second inequality, the lemma~\ref{lemma:semi-ring_2} shows that there exist pair-wise disjoint subsets $\{B_j\}_s$ such that $A,A_1,\dots,A_m$ can be expressed as
\begin{equation}
A=\bigcup_{k\in M_0}B_k,\quad A_i=\bigcup_{k\in M_i}B_k,
\end{equation}
where $M_i$ are subsets of indices in between $[1,s]$. Then the inequality is obvious since
\begin{equation}
\mu\left(\bigcup_{i=1}^nA_i\right)=\sum_{l=1}^s\mu(B_l)\ge\sum_{k\in M_0}\mu(B_k)=\mu(A).
\end{equation}
\end{proof}

It also implies that $\mu(A)\ge\mu(B)$ if $B\subseteq A$.

\begin{definition}
A measure $\mu:S_\mu\to\realR^+\cup\{0\}$ is said to be an \textbf{extension of a measure} $m:S_m\to\realR^+\cup\{0\}$ if $S_m\subseteq S_\mu$ and $\mu(A)=m(A)$ for every $A\in S_m$. 
\end{definition}

\begin{theorem}
Every measure $m$ defined on a semi-ring $S_m$ can be uniquely extended to a measure $\mu$ defined on $\mathfrak{R}(S_m)$.
\end{theorem}

\begin{proof}
Based on the lemma~\ref{lemma:extension_of_semi-ring}, the extended measure for $A=\cup_iA_i\in\mathfrak{R}(S_m)$ where $A_i\in S_m$ can be defined as
\begin{equation}
\mu(A)=\sum_{i=1}^nm(A_i).
\end{equation}
This definition is obviously nonnegative and additive. We need to prove that this measure is independent to the choice of the partition. Suppose there are two partition for $A$ that
$A=\cup_{i=1}^nB_i=\cup_{j=1}^hC_j$, then
\begin{equation}
\mu(A)=\sum_{i=1}^nm(B_i)=\sum_{ij}m(B_i\cap C_j)=\sum_{j=1}^hm(C_j).
\end{equation}
\end{proof}

\subsubsection{Complete additivity}

\begin{definition}
A measure $\mu$ is said to be \textbf{completely additive} or \textbf{$\sigma$-additive} if for countably many pair-wise disjoint sets $A_i\in S_\mu$, $A=\cup_iA_i$, the measure satisfies
\begin{equation}
\mu(A)=\sum_{i=1}^\infty\mu(A_i),
\end{equation}
where $S_\mu$ stands for a collection of sets.
\end{definition}

\begin{theorem}
Suppose $m$ is a completely additive measure defined on a semi-ring $S_m$, then the extension $\mu$ on $\mathfrak{R}(S_m)$ is completely additive 
\end{theorem}
\begin{proof}
Suppose $A\in\mathfrak{R}(S_m)$ and $A=\cup_\alpha A_\alpha$ where $A_\alpha\in S_m$ are countably many pair-wise disjointed sets, then we have $A_\alpha=\cup_kA_{\alpha k}$ and $A=\cup_{j=1}^nB_j$ where $B_j, A_{\alpha k}\in S_m$ based on the lemma~\ref{lemma:extension_of_semi-ring}, then we construct the subsets $D_{\alpha kj}=A_{\alpha k}\cap B_{j}$ and hence they are pair-wise disjointed. Since $m$ is completely additive in $S_m$, then we know $D_{\alpha kj}=A_{\alpha k}\cap B_j$ is pair-wise disjointed since it is pair-wise disjointed with respect to any one of the subscripts. 
\begin{equation}
\begin{aligned}
&\mu(A)=\sum_j\mu(B_j)=\sum_{\alpha,j,k} m(A_{\alpha k}\cap B_j)\\
&=\sum_{\alpha}m(\cup_{j,k} A_{\alpha k}\cap B_j)=\sum_\alpha \mu(A_\alpha), 
\end{aligned}
\end{equation}
The summation is commutable due to the summation is montonic and non-negative.
\end{proof}

\begin{theorem}\label{theorem:measure_compare}
If a measure $\mu$ is $\sigma$-additive, then for $A_i\in S_\mu$, and 
\begin{equation}
\mu(A)\le \sum_{i=1}^\infty\mu(A_i),\quad A\subseteq \bigcup_iA_i.
\end{equation}
\end{theorem}
\begin{proof}
It is convenient to assume that $S_\mu$ is a ring. Then we can construct the following pair-wise disjoint sets
\begin{equation}
\begin{aligned}
B_n&=(A\cap A_n)\setminus \cup_{i=1}^{n-1}A_i,\\
A&=\bigcup_jB_j,
\end{aligned}
\end{equation}
and $B_n\subseteq A_n$, then we have
\begin{equation}
\mu(A)=\sum_i\mu(B_i)\le\sum_j\mu(A_j).
\end{equation}
\end{proof}


\subsubsection{The Lebesgue extension of a measure defined on a semi-ring with unity}

The Lebesgue's definition of the measurable sets can generalize the measure to a larger classes, not only the ring of sets. The idea of this definition is that using two measurable sets to approximate the target set from upper and lower bound. 

\begin{definition}
Let $m$ is a $\sigma$-additive measure defined on semi-ring $S_m$ with a unity $E$. 
The \textbf{outer measure} of a set $A\subseteq E$ is
\begin{equation}
\mu^*=\inf\left\{\sum_nm(B_n);A\subseteq\sum_nB_n\right\},
\end{equation} 
where the lower bound is extended over all coverings of $A$ by countable (finite or denumerable) collections of sets $B_n\in S_m$. This definition can be extended to the ring $\mathfrak{R}(S_m)$ as

The outter measure of a set $A$ can also be defined as
\begin{equation}
\mu^*=\inf\left\{\sum_nm(B'_n);A\subseteq\sum_nB'_n,B'_n\in\mathfrak{R}(S_m)\right\}.
\end{equation}

The \textbf{inner measure} of a set $A\subseteq E$ is
\begin{equation}
\mu_*(A)=m(E)-\mu^*(E\setminus A).
\end{equation}
\end{definition}

The theorem~\ref{theorem:measure_compare} shows that $\mu^*(A)\ge\mu_*(A)$. And this leads to the following definition
\begin{definition}
A set $A\subseteq E$ is said to be \textbf{Lebesgue measurable} if
\begin{equation}
\mu^*(A)=\mu_*(A).
\end{equation}
The value of $\mu(A)=\mu^*(A)=\mu_*(A)$ is called \textbf{Lebesgue measure} of $A$.
\end{definition}

\begin{definition}
Given measurable spaces $(X_1,\mathfrak{S}_1)$ and $(X_2,\mathfrak{S}_2)$, a $(\mathfrak{S}_1,\mathfrak{S}_2)$-measurable functoin $F:X_1\to X_2$, and a non-negative measure $\mu$ defined on $\mathfrak{S}_1$, a \textbf{pushforward measure} $F_*\mu$ is a natural induced as
\begin{equation}
F_*\mu(A) = (\mu\circ F^{-1})(A)=\mu\left(F^{-1}(A)\right), \quad \forall A\in\mathfrak{S}_2.
\end{equation} 
\end{definition}
With this defintion, a measurable function $g$ on $X_2$ is integrable respect to the pushforward measure if and only if the composed function $g\circ F$ is integrable with respect to $\mu$, and
\begin{equation}
\int_{X_2}g(\omega_2)d(F_*\mu) = \int_{F^{-1}(X_2)}(g\circ F^{-1})(\omega_1) d\mu
\end{equation}

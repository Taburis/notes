\section{Functional Analysis}

\subsection{Measure Theory}

\subsubsection{Collections of sets}

\begin{definition}
A collection of sets $\mathfrak{S}$ is called a \textbf{semi-ring} if
\begin{enumerate}
\item $\varnothing\in\mathfrak{S}$;
\item $\forall A,B\in\mathfrak{S}:\,A\cap B\in\mathfrak{S}$;
\item If $A, A_1\in\mathfrak{R}$ and $A_1\subseteq A$, then $\exists A_i\in\mathfrak{S},i=2,\dots,n$ so that
\begin{equation}
A=\bigcup^n_{i=1} A_i,\quad \forall i\ne j:A_i\cap A_j=\varnothing.
\end{equation} 
\end{enumerate}
\end{definition}


\begin{definition}
A collection of sets $\mathfrak{R}$ is called a \textbf{ring} if $\forall A,B\in\mathfrak{R}$:
\begin{enumerate}
\item $A\cap B\in\mathfrak{R}$,
\item $A\triangle B\in\mathfrak{R}$.
\end{enumerate}
\end{definition}

A ring $\mathfrak{R}$ is also closed for the operation union and difference since
\begin{equation}
\begin{aligned}
A\cup B&=(A\triangle B)\triangle(A\cap B)\\
A\setminus B&=A\cap(A\triangle B),
\end{aligned}
\end{equation}
and it implies that $\varnothing\in\mathfrak{R}$ since $A\setminus A=\varnothing$.

\begin{lemma}
Given a sequence of pair-wise disjoint subsets $A_1,\dots,A_m\in\mathfrak{S}$, then there exists another sequence $A_{m+1},\dots,A_n\in\mathfrak{S}$ such that
\begin{equation}
A=\bigcup_{i=1}^nA_i,\quad A_i\cap A_j=\varnothing\text{ for }i\ne j. 
\end{equation}
\end{lemma}

\begin{proof}
The proof is based on the induction method, and the $n=1$ case valid by the definition of semi-ring. Suppose the lemma satisfies for $m=M$, and we will prove it valid for $M+1$ case. In fact, let $S_{M}=\sum_{i=1}^MA_i$, then $S_M\subseteq A$ which means that $\exists B_j,j=1,\dots,B_s$ such that
\begin{equation}
A=S_M\cup\sum_{j=1}^sB_j,\quad B_i\cap B_j\text{ for }i\ne j.
\end{equation}
Now, we set $B_{1,j}=A_{M+1}\cap B_j$, then we have a pair-wise disjointed sequence of $B_{k,j}$ such that
\begin{equation}
B_i=\bigcup_{k=1}^{n_i}B_{k,i},
\end{equation}
Then we got finally
\begin{equation}
A=\left(\bigcup_{i=1}^{M+1}A_i\right)\cup\left(\bigcup_{j=1}^s\bigcup_{k=2}^{n_j}B_{k,j}\right).
\end{equation}
\end{proof}



\begin{definition}
Given measurable spaces $(X_1,\mathfrak{S}_1)$ and $(X_2,\mathfrak{S}_2)$, a $(\mathfrak{S}_1,\mathfrak{S}_2)$-measurable functoin $F:X_1\to X_2$, and a non-negative measure $\mu$ defined on $\mathfrak{S}_1$, a \textbf{pushforward measure} $F_*\mu$ is a natural induced as
\begin{equation}
F_*\mu(A) = (\mu\circ F^{-1})(A)=\mu\left(F^{-1}(A)\right), \quad \forall A\in\mathfrak{S}_2.
\end{equation} 
\end{definition}
With this defintion, a measurable function $g$ on $X_2$ is integrable respect to the pushforward measure if and only if the composed function $g\circ F$ is integrable with respect to $\mu$, and
\begin{equation}
\int_{X_2}g(\omega_2)d(F_*\mu) = \int_{F^{-1}(X_2)}(g\circ F^{-1})(\omega_1) d\mu
\end{equation}
